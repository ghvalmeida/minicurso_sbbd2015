%
% Sample SBC book chapter
%
% This is a public-domain file.
%
% Charset: ISO8859-1 (latin-1) ������
%
\documentclass{SBCbookchapter}
\usepackage[latin1]{inputenc}
\usepackage[T1]{fontenc}
\usepackage[brazilian,english]{babel}
\usepackage{graphicx}
\usepackage{algorithm}
\usepackage{algorithmic}
\usepackage{amsmath}
\usepackage{listings}
\usepackage{color}
\usepackage{breakcites}

\makeatletter
\newcommand{\newalgname}[1]{%
  \renewcommand{\ALG@name}{#1}%
}
\newalgname{Algoritmo}% All algorithms will be called "Algorithme"
\renewcommand{\listalgorithmname}{Lista de \ALG@name s}
\makeatother

\renewcommand{\algorithmicrequire}{\textbf{Input:}}
\renewcommand{\algorithmicensure}{\textbf{Output:}}
\renewcommand{\algorithmicif}{\textbf{se}}
\renewcommand{\algorithmicthen}{\textbf{ent�o}}
\renewcommand{\algorithmicdo}{\textbf{fa�a}}
\renewcommand{\algorithmicend}{\textbf{fim}}
\renewcommand{\algorithmicfor}{\textbf{para}}

\definecolor{dkgreen}{rgb}{0,0.6,0}
\definecolor{gray}{rgb}{0.5,0.5,0.5}
\definecolor{mauve}{rgb}{0.58,0,0.82}

\lstdefinestyle{MySQLStyle}{
  frame=tb,
  language=SQL,
  aboveskip=3mm,
  belowskip=3mm,
  showstringspaces=false,
  columns=flexible,
  basicstyle={\small\ttfamily},
  numbers=none,
  numberstyle=\tiny\color{gray},
  keywordstyle=\color{blue},
  commentstyle=\color{dkgreen},
  stringstyle=\color{mauve},
  breaklines=true,
  breakatwhitespace=true,
  tabsize=3
}

\lstdefinestyle{MyPythonStyle}{
  frame=tb,
  language=Python,
  aboveskip=3mm,
  belowskip=3mm,
  showstringspaces=false,
  columns=flexible,
  basicstyle={\small\ttfamily},
  numbers=none,
  numberstyle=\tiny\color{gray},
  keywordstyle=\color{blue},
  commentstyle=\color{dkgreen},
  stringstyle=\color{mauve},
  breaklines=true,
  breakatwhitespace=true,
  tabsize=3
}

\lstset{language=Python,frame=tb}
\lstset{language=SQL,frame=ltrb}

\author{Victor Teixeira de Almeida e Vitor Alc�ntara Batista}
\title{Tecnologias para Gerenciamento de Dados na Era do Big Data}

\begin{document}
\maketitle

\selectlanguage{brazilian}

\begin{abstract}
%\begin{otherlanguage}{english}
This chapter is an overview of big data management, and
intends to explore and diferentiate several recent technologies. A
classic problem of the database community will be used as a background for the 
examples given throughout this course: triangle counting on graphs. This 
problem has been chosen because it is being extensively used to identify 
the importance of individuals on social networks. 
Also, since it can be described by an algorithm that is simple to understand and 
yet complex to execute in terms of performance, the differences between technologies
in design and performance will be easily demonstrated. 
%\end{otherlanguage}
\end{abstract}

\begin{resumo}
Este cap�tulo pretende explorar e diferenciar de forma introdut�ria diversas 
tecnologias recentes
para gerenciamento de dados na era do big data. Ser� utilizado como pano de fundo para os
exemplos um problema cl�ssico da comunidade de bancos de dados: a contagem de tri�ngulos
em grafos. Ele foi escolhido por ser um problema atual e pr�tico, frequentemente 
utilizado para identificar a import�ncia de indiv�duos em redes sociais. Al�m disso, ele � de
f�cil representa��o e alta complexidade de execu��o. Atrav�s do seu uso, � poss�vel demonstrar
as diferen�as entre as tecnologias em termos de expressividade e desempenho. 
\end{resumo}


\section{Introdução}
Esse artigo blablabal

\section{Contagem de triângulos}

Nesta seção, iremos descrever o problema da contagem de triângulos, que irá nos acompanhar
ao longo deste minicurso. Este problema é extremamente interessante para avaliação de 
tecnologias uma vez que é de simples descrição e implementação, logo didádico, e complexo
em termos de execução, desempenho.

O problema, como o próprio nome diz, consiste em contar triângulos em um grafo, ou seja,
contar os subgrafos $t_i$ de um grafo $G$ contendo 3 diferentes vértices conectados entre si 
(triângulos). 

Uma das grandes aplicações da contagem de triângulos é o cálculo do coeficiente de agrupamento 
de um nó em um grafo ou do grafo como um todo. Esta métrica que possui diversas aplicações práticas 
em análises de redes sociais. O coeficiente de agrupamento de um nó $v$ expressa a probabilidade de 
dois nós vizinhos a $v$ serem também vizinhos entre si. Para o grafo $G$ como um todo, o coeficiente 
de agrupamento é a média dos coeficientes de cada vértice do grafo; altos valores para este coeficiente 
significam uma comunidade coesa (\emph{small world community}).

Implementações de algoritmos eficientes para este problema abundam na literatura. O Algoritmo 
\ref{alg:inmemorytrianglecounting}, adapdado de \cite{Chu2012} para retornar somente a contagem de 
triângulos, apresenta uma execução eficiente para o problema, e é a base para as principais 
implementações de algoritmos que assumem que os dados cabem na memória. 

\begin{algorithm}
\caption{Algoritmo para contagem de triângulos em memória}
\label{alg:inmemorytrianglecounting}
\begin{algorithmic}[1]
    \REQUIRE Grafo $G = (V, E)$
    \ENSURE $c$, a contagem de triângulos em $G$
    \STATE $c \leftarrow 0$
    \FOR{cada $v \in V$}
        \FOR{cada $u \in adj_G(v)$, dado que $u>v$}
            \FOR{cada $w \in ( adj_G(v) \cap adj_G(u)$, dado que $w>u$}
                \STATE{$c \leftarrow c + 1$}
            \ENDFOR
        \ENDFOR
    \ENDFOR 
    \STATE{return($c$)}
\end{algorithmic}
\end{algorithm}

Este algoritmo começa por inicializar a variável $c$ de contagem de triângulos com $0$. Então, para
todos os vértices do grafo (nomeados $v$), o algoritmo tenta resgatar um vértice que seja adjacente a 
$v$, nomeado de $u$, e outro vértice que seja ao mesmo tempo adjacente a $v$ e a $u$, nomeado de $w$.
Cada vez que esses três vértices conectados por arestas são encontrados, o algoritmmo incrementa o 
contador de triângulos $c$. Há mais um ponto a ser explicado aqui que é um teste para remover duplicatas
que garante que $u>v$ e $w>u$, assumindo que os vértices possuam identificadores únicos no domínio dos
números naturais, por exemplo. A complexidade deste algoritmo depende da implementação da busca pelos 
vértices $u$ e $w$ nas listas de adjacências de $v$ e $u$, respectivamente.

Vamos agora mostrar como definir este algoritmo em SQL.

\begin{lstlisting}[style=MySQLStyle]
SELECT follower 
FROM TWITTER;
\end{lstlisting}

\section{Bancos de dados relacionais}
\label{sec:relacional}

Bancos de dados relacionais s�o sistemas que implementam o modelo de dados
relacional proposto por Codd na d�cada de 70 \cite{Codd1970}, rapidamente
implementado por um sistema chamado System R da IBM \cite{Astrahan1976}.
A principal ideia � a de representar os dados em forma de rela��es (tabelas)
e as opera��es s�o definidas a partir de uma �lgebra: a �lgebra relacional.
O sistema recebe comandos atrav�s de uma linguagem declarativa (SQL) 
\cite{Chamberlin1974}, os converte em operadores da �lgebra relacional
para ent�o execut�-los nos dados persistentes nas rela��es.

\begin{figure}[!htbp]
        \centering
        \includegraphics[width=0.75\linewidth]{./colunar_repr_tabela.png}
        \caption{Representa��o de uma tabela em um banco de dados relacional.}
        \label{fig:tabular}
\end{figure}

A Figura \ref{fig:tabular} mostra a representa��o de uma rela��o contendo 
informa��es sobre indiv�duos em um formato de tabela. Esta rela��o possui
os seguintes atributos do indiv�duo : (i) \emph{fname}, o nome; (ii) 
\emph{lname}, o sobrenome; (iii) \emph{gender}, o sexo, \emph{m} para 
masculino e \emph{f} para feminino; (iv) \emph{city},
a cidade onde nasceu; (v) \emph{country}, o pa�s onde nasceu; e 
(vi) \emph{birthday}, a data de nascimento.

Os principais operadores da �lgebra relacional s�o: proje��o ($\Pi$),
sele��o ($\sigma$), jun��o ($\bowtie$). Sejam duas rela��es $R$ e $S$
com atributos $r_1, \ldots, r_m$ e $s_1, \ldots, s_n$, representadas
por $R(r_1, \ldots, r_m)$ e $S(s_1, \ldots, s_n)$. Uma proje��o em $R$
recebe como argumento uma lista de atributos subconjunto dos atributos
de $R$ e retorna a rela��o contendo somente os atributos desta lista.
A proje��o � representada por $\Pi_{lista~de~atributos}(R)$. Uma sele��o
em $R$ recebe como argumentos uma condi��o em forma de predicado e 
retorna todas as tuplas em $R$ que satisfazem o predicado. A sele��o 
� representada por $\sigma_{predicado} (R)$. A jun��o � um operador
bin�rio e � aplicado sobre duas rela��es $R$ e $S$. Ele recebe como argumento
uma condi��o em forma de predicado (usualmente a igualdade entre dois
atributos das rela��es $R$ e $S$), e retorna a rela��o correspondente
ao produto cartesiano das duas rela��es $R$ e $S$ cujas tuplas satisfazem
o predicado da jun��o. A jun��o � representada por $R \bowtie_{predicado} S$. 

A linguagem SQL expressa de forma declarativa opera��es sobre os dados
armazenados nas rela��es a partir da �lgebra relacional. Uma vers�o simplificada
da linguagem SQL � mostrada a seguir:

\begin{lstlisting}[style=MySQLStyle]
SELECT <lista de atributos>
FROM   <lista de tabelas>
WHERE  <predicado>;
\end{lstlisting}

Ap�s o sistema receber uma consulta na linguagem SQL, esta deve ser expressada
segundo operadores da �lgebra relacional da forma mais eficiente poss�vel. Um 
otimizador de consultas � o respons�vel por esta tradu��o. A ordem dos operadores
influencia no resultado; normalmente sele��es ($\sigma$) s�o as primeiras a serem
executadas, pois reduzem o tamanho das rela��es resultantes. A ordem das jun��es
tamb�m � de extrema import�ncia; uma jun��o entre as rela��es $R$, $S$ e $T$ pode
ser executada nas seguintes ordens: (i) $R \bowtie S \bowtie T$, (ii) 
$R \bowtie T \bowtie S$ e (iii) $S \bowtie T \bowtie R$. Adicionalmente, existem
in�meros algoritmos para a execu��o eficiente das jun��es que devem coexistir no
sistema e serem utilizados em circunst�ncias em que s�o os mais eficientes 
\cite{Mishra1992}.

O problema da contagem de tri�ngulos da Se��o \ref{sec:triangulos} pode ser expresso na 
seguinte consulta SQL, assumindo que h� uma rela��o chamada Twitter com os attributos
\emph{follower} e \emph{followee} contendo n�meros inteiros de identificadores de usu�rios
representando relacionamentos na rede social, onde \emph{follower} segue \emph{followee}.

\begin{lstlisting}[style=MySQLStyle]
SELECT COUNT(*) 
FROM   TWITTER R, TWITTER, S, TWITTER T
WHERE  R.followee = S.follower AND
       S.followee = T.follower AND
       T.followee = R.follower AND
       R.follower > S.follower AND
       S.follower > T.follower;
\end{lstlisting}

Uma ordem de execu��o desta consulta, expressa em operadores da �lgebra relacional � 
\begin{multline}
\sigma_{S.follower > T.follower}(\sigma_{R.follower > S.follower}(( R \bowtie_{R.followee = S.follower} S ) \\
\bowtie_{(S.followee = T.follower) \land (T.followee = R.follower)} T))
\end{multline}

Este plano de execu��o ir� seguir os seguintes passos:

\begin{enumerate}
\item Primeira jun��o entre as tabelas $R$ e $S$ com predicado $R.followee = S.follower$
\item Sele��o com predicado $R.follower > S.follower$ aplicado ao resultado anterior
\item Jun��o do resultado anterior com a tabela $T$ com predicado $(S.followee = T.follower) \land (T.followee = R.follower)$
\item Sele��o com predicado $S.follower > T.follower$
\end{enumerate}



\section{Bancos de dados colunares} \label{colunar}
A idéia por traz dos bancos de dados colunares não é nova. Uma das primeiras propostas de organizar os dados de um banco de dados por colunas, em vez da tradicional representação por linhas dos bancos de dados relacionais, apareceu em 1969 \cite{Estabrook1969327}. De forma resumida, enquanto um banco de dados relacional armazena cada registro (tupla) em um espaço contínuo do disco (ou memória), os bancos de dados colunares armazenam cada coluna em espaços contínuos. Para isso, cada coluna de uma tabela (ou relação), é dividia em um índice com os valores distintos ordenados e um vetor com a codificação dos valores de cada tupla na mesma sequência em que aparecem na tabela. A Figura \ref{fig:tabular} mostra a representação tradicional de uma tabela num banco de dados relacional e a Figura \ref{fig:colunar} mostra como a primeira coluna da relação é organizada em um banco de dados colunar.

\begin{figure}
	\centering
	\includegraphics[width=\linewidth]{./Representacao_colunar.jpg}
	\caption{Organização da coluna fname em um banco de dados colunar.}
	\label{fig:colunar}
\end{figure}

A grande vantagem da representação colunar é a compressão de dados, que ocorre, principalmente, pela codificação com um numero mínimo de bits dos valores de cada registro, baseada no dicionário. Com os dados comprimidos, menos Bytes trafegam entre o disco e a memória principal, tornando operações de consulta e agregação mais rápidas.

Para explicar melhor, imaginemos que a coluna \textbf{fname} tenha 100 caracteres representados por 1 Byte cada. Imaginemos também que a tabela em questão trata-se da lista de todos os cidadãos brasileiros, ou seja, teremos 200 milhões de registros. Logo, o espaço de armazenamento no modelo tradicional é de $ 200*10^6 * 100 Bytes \cong 18,62 GBytes  $. No armazenamento em formato de colunas, os dados são armazenados em um dicionário que contém uma entrada para cada valor distinto. Imaginemos que há 10 mil primeiro nomes distintos no Brasil. Com isso, para o dicionário, precisamos de $ 10*10^3 * 100 Bytes \cong 1 MBytes  $. Já o vetor de valores conterá os 200 milhões de registros, mas codificados pela quantidade mínima de bits necessários para se codificar os 10 mil valores distintos $ log_2 10000 \cong 13,2  $, ou seja, 14 bits. Nesse caso, o vetor de valores terá $ 200*10^6 * 14 bits \cong 2,6 GBytes  $. Nesse simples exemplo, podemos ver um fator de compressão de $ 18,62 / (0,001 + 2,6) \cong 7,2 $ vezes.

A recuperação de um determinado registro (tupla) passa por um acesso direto à posição dele em cada uma dos vetores de valor das colunas da tabela e mais um acesso em cada dicionário para traduzir a codificação no valor original.

O grande problema nessa representação são operações de remoção e inserção, que usualmente causam uma reorganização do dicionário e consequentemente uma reorganização de todos os dados da coluna cujo dicionário foi reordenado. Por isso, o uso desse tipo de tecnologia deve ser preferível em conjuntos de dados cuja operação predominante seja de consulta. Plattner \cite{plattner2009common} alega que os bancos de dados corporativos possuem uma carga predominantemente de consulta e com isso, pode-se unificar os repositórios OLTP e OLAP na mesma estrutura. 

Os sistemas modernos que utilizam a representação colunar lançam mão de diversas estratégias para minimizar esses efeitos de reorganização dos dicionários. O SAP HANA, por exemplo, divide os dados em principal e diferencial \cite{plattner2012memory}. Novos registros e atualizações de valores são incluídos no conjunto diferencial, que é mantido em tamanho pequeno. As operações de consulta juntam os dados do conjunto principal com o conjunto diferencial. Embora a busca no conjunto diferencial seja ineficiente, ela é realizada sobre um conjunto pequeno de dados. De tempos em tempos, os conjuntos são mesclados resultando em um novo conjunto principal e um conjunto diferencial praticamente vazio, que conterá apenas as operações realizadas durante a operação de mesclagem.

Atualmente, os principais fornecedores de bancos de dados relacionais também fornecem soluções de armazenamento colunar, como Microsoft, Oracle, IBM, SAP e outros.  





\section{Bancos de dados em memória}

Sistemas de banco de dados em memória são aqueles em que a fonte primária dos dados reside em memória principal (RAM). Esses dados tem, usualmente, cópia em disco, para eventuais falhas no hardware, como falta de energia. Embora os sistemas de bancos de dados tradicionais também mantém algum dado em memória como forma de fazer \textit{cache}, a principal diferença é que nos bancos de dados em memória os dados a fonte de dado primária (ou principal) está armazenada na memória RAM, enquanto nos tradicionais, está armazenada em disco \cite{garcia1992main}.

Esse tipo de tecnologia ganhou força nos últimos anos devido aos avanços nas arquiteturas de hardware que, atualmente, permitem sistemas com Terabytes de memória RAM compartilhadas entre vários processadores. Além disso, houve um barateamento enorme no custo desses equipamentos, o que tornou viável os sistemas de banco de dados em memória. 

Como o acesso à memória principal chega a ser 1000 vezes mais rápido que os discos modernos como SSD (disco de estádo sólido), esse tipo de tecnologia é bem convidativo. O leitor mais atento pode se perguntar: E se um sistema de banco de dados tradicional tiver um \textit{cache} grande o suficiente para caber todo o volume de dados, qual a diferença? Ainda sim esses sistemas são projetados de forma não ótima para uso da memória. Por exemplo, será necessário consultar um gerenciador do \textit{cache} toda vez que for acessar o dado e os índices estão em estruturas onde o acesso não é imediato.

Atualmente, praticamente todos os grandes fornecedores de soluções tradicionais de bancos de dados relacionais também possuem versões em memória de seus produtos, como Microsoft, Oracle e IBM, mas também há fornecedores especializados nesse tipo de solução. Alguns desses produtos também possuem características de bancos de dados colunares, que descrevemos na seção \ref{colunar}.

A seguir discutiremos algumas questões relacionadas ao projeto de sistemas de banco de dados em memória.

\subsection{Controle de concorrência}

\subsection{Processamento de transações}
\subsection{Método de acesso}
\subsection{Recuperação de falhas}






\section{Bancos de dados paralelos}

Parallel databases \cite{Dewitt1992}

Shared-nothing \cite{Stonebraker1986}

Aqui vamos falar sobre bancos de dados paralelos.

\section{Hadoop}

\subsection{Breve histórico}

O problema era simples: como criar um índice para uma máquina de busca de toda a Internet? Foi com esse 
desafio que Mike Cafarella e Doug Cutting resolveram desenvolver o Apache Nutch. Rapidamente o 
\textit{crawler} e a máquina de busca ficaram prontos, mas eles perceberam que a arquitetura não 
escalaria para criar um índice de mais de um bilhão de páginas da Internet. Na mesma época, a equipe 
do Google publicou um artigo que explicava a arquitetura do GFS (Google FileSystem) 
\cite{ghemawat2003google}, que era um sistema de arquivos distribuído usado em sua máquina de busca. 
Doug e Mike decidiram criar uma implementação \textit{open source} dessa arquitetura e a chamaram 
de NDFS (Nutch Distributed FileSystem).

Em 2004, a equipe do Google publicou novo artigo agora detalhando como era possível criar um índice de 
toda a Internet usando o mecanismo de processamento paralelo denominado MapReduce \cite{dean2008mapreduce}. 
Com base nesse trabalho, os desenvolvedores do Nutch migraram a maior parte de seus algoritmos para 
executar sobre o MapReduce e o NDFS. Mais tarde, Doug Cutting foi trabalhar no Yahoo! liderando uma 
equipe que construiu a nova geração de máquina de busca deles. Depois, o NDFS (posteriormente chamado
de HDFS ou \textit{Hadoop Distributed Filesystem}) e o MapReduce tornaram-se 
um projeto da Apache Software Foundation sob o nome de Apache Hadoop.

Desde então, o Hadoop tem sido usado mundialmente para processar enormes quantidades de dados. Vários 
frameworks foram construídos para executar usando a sua infraestrutura, como veremos a seguir. 
Diversos fornecedores criaram suas próprias distribuições do Hadoop, como Microsoft, IBM, EMC, Oracle e 
outras empresas especializadas como Cloudera e Hortonworks.

\subsection{Funcionamento do HDFS}

O HDFS, como mencionado acima, é um sistema de arquivos distribuídos, projetado para armazenar arquivos 
muito grandes\footnote{Atualmente há instâncias do HDFS armazenando PetaBytes de dados.} executando sobre 
hardware commodity. 

Assim como em qualquer sistema de arquivos, um arquivo é dividido em \textbf{blocos} de dados. Enquanto 
tipicamente um sistema de arquivos tradicional armazena dados em pequenos blocos (e.g. 512 bytes ou 1 kbyte), 
o HDFS usa, por padrão, blocos de 64MB. Isso torna o HDFS ineficiente para uso em arquivos muito pequenos 
e numerosos. Para garantir disponibilidade e leitura em paralelo, cada um dos blocos é replicado em um dos 
nós de um \textit{cluster} HDFS. Quando um disco ou um dos nós do \textit{cluster} falha, além do bloco 
poder ser lido de outro nó, o sistema de arquivos automaticamente recria os blocos presentes naquele disco 
em outros nós do \textit{cluster}.

\begin{figure}
	\centering
	\includegraphics[width=\linewidth]{./Arquitetura_HDFS.jpg}
	\caption{Visão geral da arquitetura do HDFS}
	\label{fig:hdfs_arch}
\end{figure}

A arquitetura de um \textit{cluster} HDFS divide-se em \textit{NameNode} e \textit{DataNode}. O primeiro armazena um índice de arquivos e de seus blocos e o segundo armazena os dados (blocos). Um cliente que queira ler arquivos no HDFS, primeiro consulta o \textit{NameNode}, que então diz de quais nós do cluster os blocos serão lidos, garantindo um balanceamento de carga na leitura. Arquivos criados no HDFS só podem ser modificados anexando conteúdo no final. Não pode-se modificar blocos já escritos. A falha do \textit{DataNode} implica na indisponibilidade de todo o HDFS e, por esse motivo, é importante mantê-lo resiliente a falha com mecanismos de redundância. Uma visão geral dessa arquitetura pode ser vista na figura \ref{fig:hdfs_arch}.

\subsection{O ecossistema Hadoop}
Atualmente o Hadoop conta com um ecossistema com diversos \textit{frameworks}. Uma lista, não exaustiva, de alguns dos principais projetos que compõem o ecossistema pode ser vista abaixo.
\begin{itemize}
	\item Ambari: Uma ferramenta web para aprovisionamento e gerenciamento de um cluster Hadoop e de diversos de seus componentes.
	\item HBase: Um banco de dados relacional e colunar que utiliza a infraestrutura do Hadoop como mecanismo de armazenamento.
	\item Hive: Uma infraestrutura de armazém de dados com suporte a sumarização de dados e consultas.
	\item Pig: Uma linguagem de alto nível para fluxo de dados e um \textit{framework} de execução de computação distribuída. 
	\item Spark: Uma \textit{engine} rápida e de propósito geral para processamento de dados em memória baseados nos dados do HDFS. O Spark oferece um modelo de programação simples e poderoso para executar uma enorme gama de atividades como ETL, aprendizagem de máquina, processamento contínuo de dados, processamento de grafos, etc.
	\item Sqoop: uma ferramenta para transferência massiva de dados entre bancos de dados relacionais e o HDFS.
	\item Mahout: Um conjunto de bibliotecas para executar algoritmos de aprendizagem de máquina e mineração de dados. Os coordenadores do projeto decidiram mover a implementação dos algoritmos de MapReduce para o Spark.
\end{itemize}

\subsection{MapReduce}
O Hadoop MapReduce é um \textit{framework} para facilitar a escrita de programas de computador para processar uma enorme quantidade de dados de forma paralela, distribuída e resiliente a falhas. Os dados de entrada para um \textit{Job} MapReduce, por estarem armazenados no HDFS, são também processados de forma distribuída, aproveitando dos dados disponíveis localmente em um nó do \textit{cluster}. Na \ref{fig:mapreduce} podemos ver um desenho esquemático de um \textit{Job} MapReduce, que  é, de forma resumida, composto por duas fases:
\begin{enumerate}
	\item Map - quando os dados são processados e produzem saídas como tuplas no formato (Chave, Valor); 
	\item Reduce - quando as tuplas com mesma chave são agrupadas para alguma atividade de agregação.  
\end{enumerate}

\begin{figure}
	\centering
	\includegraphics[width=\linewidth]{./Job_Mapreduce.jpg}
	\caption{Um \textit{Job} MapReduce}
	\label{fig:mapreduce}
\end{figure}

Um exemplo bem simples para entender o MapReduce é um Job para contar a ocorrência de cada palavra em um texto. Na fase de Map, cada linha lida do arquivo é dividida em suas palavras que produzem uma saída (PalavraA, 1). Note que se uma palavra aparece duas vezes na mesma linha duas tuplas idênticas serão produzidas. Na faze de Reduce, as tuplas com mesma chave serão agrupadas e os valores serão somados. 

Em geral, o programador não precisa se preocupar com comunicação de dados, tratamento de concorrência e eventuais falhas em algum nó que está processando um determinado \textit{Job}. Esse é um dos grandes diferenciais do Hadoop MapReduce. 

Uma possível solução para contagem de triângulos utilizando MapReduce, baseada na proposta feita por Suri e Sergei \cite{suri2011counting} envolve o encadeamento de 3 \textit{Jobs} MapReduce encadeados. O primeiro constrói um conjunto de todas as tríades (par de arestas que compartilham um vértice) de um grafo. Essas tríades, juntamente com as arestas originais são gerados como linhas, mas com um atributo identificador dizendo se é uma aresta original do arquivo ou se foi gerada pelo primeiro passo. Essa saída serve de entrada para o segundo \textit{Job}, que particiona XXX. Uma dessas partições conterá $n$ triângulos se ela contém uma aresta original e $n$ tríades. Um terceiro \textit{Job} counta o número de triângulos gerados pelo passo anterior e produz o resultado final. Vamos olhar um exemplo para melhor entendimento.

Dada o grafo não direcionado de entrada descrita na tabela \ref{grafoExemplo}, onde os números identificam os nós e cada linha uma aresta que liga esses vértices, podemos identificar 3 triângulos: (Bernardo, Chico, Renato), (Renato, Chico, Roberto) e (André, Bianca, Marcelo).

\begin{table}[]
\centering
\caption{Exemplo de grafo de entrada.}
\label{grafoExemplo}
\begin{tabular}{cc}
\hline
{\bf Origem} & {\bf Destino} \\ \hline
André        & Bernardo      \\ \hline
André        & Marcelo       \\ \hline
André        & Bianca        \\ \hline
Bernardo     & Chico         \\ \hline
Bernardo     & Renato        \\ \hline
Chico        & Renato        \\ \hline
Chico        & Roberto       \\ \hline
Chico        & Livia         \\ \hline
Marcelo      & Bianca        \\ \hline
Roberto      & Renato        \\ \hline      
\end{tabular}
\end{table}

Após a etapa de Map do primeiro \textit{Job}, são gerados um conjunto de (chave, valor) conforme a tabela \ref{map1}. Note que a chave é escolhida como o vértice de menor valor.

\begin{table}[]
\centering
\caption{Resultado da etapa de Map do primeiro \textit{Job}.}
\label{map1}
\begin{tabular}{cc}

\hline
{\bf Chave} & {\bf Valor (arest)}      \\ \hline
André       & André, Marcelo   \\ \hline
André       & André, Bianca    \\ \hline
André       & André, Bernardo  \\ \hline
Marcelo     & Marcelo, Bianca  \\ \hline
Bernardo    & Bernardo, Chico  \\ \hline
Bernardo    & Bernardo, Renato \\ \hline
Chico       & Chico, Roberto   \\ \hline
Chico       & Chico, Livia     \\ \hline
Chico       & Chico, Renato    \\ \hline
Roberto     & Roberto, Renato  \\ \hline     
\end{tabular}
\end{table}

A etapa de Reduce do primeiro \textit{Job} é a mais importante. Nessa etapa, para cada chave da tabela \ref{map1}, são gerados as arestas originais (valor na tabela) e as tríades baseadas nessas arestas. O resultado, já ordenado, pode ser visto na tabela \ref{reduce1}.

\begin{table}[]
\centering
\caption{Resultado da etapa de Reduce do primeiro \textit{Job}.}
\label{reduce1}
\begin{tabular}{ccc}
\hline
{\bf Chave}                                                    & {\bf Gerado} & {\bf Tríades}                                                                         \\ \hline
\begin{tabular}[c]{@{}c@{}}André,\\   Bernardo\end{tabular}    & 0            &                                                                                       \\ \hline
\begin{tabular}[c]{@{}c@{}}André,\\   Marcelo\end{tabular}     & 0            &                                                                                       \\ \hline
\begin{tabular}[c]{@{}c@{}}André,\\   Bianca\end{tabular}      & 0            &                                                                                       \\ \hline
\begin{tabular}[c]{@{}c@{}}Bernardo,\\   Chico\end{tabular}    & 0            &                                                                                       \\ \hline
\begin{tabular}[c]{@{}c@{}}Bernardo,\\   Bianca\end{tabular}   & 1            & \begin{tabular}[c]{@{}c@{}}\{André, Bernardo\}, \{André,\\   Bianca\}\end{tabular}    \\ \hline
\begin{tabular}[c]{@{}c@{}}Bernardo,\\   Renato\end{tabular}   & 0            &                                                                                       \\ \hline
\begin{tabular}[c]{@{}c@{}}Bernardo, \\   Marcelo\end{tabular} & 1            & \begin{tabular}[c]{@{}c@{}}\{André, Bernardo\}, \{André,\\   Marcelo\}\end{tabular}   \\ \hline
\begin{tabular}[c]{@{}c@{}}Chico,\\   Livia\end{tabular}       & 0            &                                                                                       \\ \hline
\begin{tabular}[c]{@{}c@{}}Chico,\\   Roberto\end{tabular}     & 0            &                                                                                       \\ \hline
\begin{tabular}[c]{@{}c@{}}Chico,\\   Renato\end{tabular}      & 1            & \begin{tabular}[c]{@{}c@{}}\{Bernardo, Chico\}, \{Bernardo,\\   Renato\}\end{tabular} \\ \hline
\begin{tabular}[c]{@{}c@{}}Chico,\\   Renato\end{tabular}      & 0            &                                                                                       \\ \hline
\begin{tabular}[c]{@{}c@{}}Livia,\\   Roberto\end{tabular}     & 1            & \{Chico, Livia\}, \{Chico, Roberto\}                                                  \\ \hline
\begin{tabular}[c]{@{}c@{}}Livia,\\   Renato\end{tabular}      & 1            & \{Chico, Livia\}, \{Chico, Renato\}                                                   \\ \hline
\begin{tabular}[c]{@{}c@{}}Marcelo,\\   Bianca\end{tabular}    & 1            & \begin{tabular}[c]{@{}c@{}}\{André, Marcelo\}, \{André,\\   Bianca\}\end{tabular}     \\ \hline
\begin{tabular}[c]{@{}c@{}}Marcelo,\\   Bianca\end{tabular}    & 0            &                                                                                       \\ \hline
\begin{tabular}[c]{@{}c@{}}Roberto,\\   Renato\end{tabular}    & 1            & \begin{tabular}[c]{@{}c@{}}\{Chico, Roberto\}, \{Chico,\\   Renato\}\end{tabular}     \\ \hline
\begin{tabular}[c]{@{}c@{}}Roberto,\\   Renato\end{tabular}    & 0            &                                                                                       \\ \hline
\end{tabular}
\end{table}

O segundo \textit{Job} tem apenas a etapa de Reduce e toma de entrada o resultado produzido na tabela \ref{reduce1}. Para cada grupo de chaves (arestas), é verificado se há uma aresta original (Gerado = 0). Se sim, é verificada em quantas tríades essa aresta fecha um triângulo, emitindo uma contagem parcial para cada chave. Pelo nosso exemplo, podemos ver que isso ocorre nas chaves (Chico,Renato), (Lívia,Renato) e (Marcelo, Bianca), totalizando nossos 3 triângulos.

A última etapa (terceiro \textit{Job}) é uma tarefa trivial de somar as contagens parciais emitidas pelo passo anterior.

These triads and the original edges are emitted as rows by the first MR job, with a field added to distinguish the two. Note that the output of the first job can be quite large, especially in a dense graph. Such output is consumed by the second MR job, which partitions the rows by either the unclosed edge, if the row is a triad, or the original edge. A partition has n triangles if it contains an original edge and n triads. A third, trivial MR job counts the triangles produced by the second job, to produce the final result.

\subsection{Pig}
nessa seção vou explicar o mapreduce e como resolver o problema do triangulo

\section{Apache Spark}
O Apache Spark é uma plataforma para computação distribuída que foi projetada para ser de propósito geral e muito eficiente \cite{karau2015learning}. A principal diferença em relação ao MapReduce é que toda a computação é feita e armazenada em memória, sem necessidade de salvar em disco resultados intermediários. 

A unidade básica de dados do Spark é o \textit{Resilient Distributed Dataset} (RDD). O conceito é semelhante ao bloco de dados do HDFS, mas trata-se de coleções de dados que estão na memória RAM dos nós do \textit{cluster}. Na prática, o Spark carrega os dados de um bloco do HDFS na memória RAM do nó em que o bloco está. Os programas Spark fazem dois tipos de operação com um RDD:
\begin{itemize}
	\item \textbf{Transformações}: Transformam um RDD em outro RDD. Entre as transformações mais comuns podemos encontrar as operações de Map e ReduceByKey (mesmo conceito do MapReduce), ordenação e operações de conjuntos, como união, interseção e diferença.
	\item \textbf{Ações}: Produzem algum resultado a partir de um RDD. Ações típicas consistem em enumerar alguma quantidade de itens de um RDD, contar e somar. 
\end{itemize}

O Spark utiliza o conceito de "execução preguiçosa", para que só quando realmente um resultado tenha de ser produzido (uma \textbf{ação} é executada), e toda a sequência de passos necessária é conhecida, o Spark realmente lê os dados e faz cálculos na memória. Com isso, o motor de execução do Spark consegue otimizar tudo que será executado, escolhendo os melhores nós, a melhor sequência, etc.

O Apache Spark também vem com um conjunto de bibliotecas com algoritmos para aprendizado de máquina, grafos, fluxo contínuo de dados e SQL. Algumas dessas vermos a seguir.

\subsection{Utilizando o console python}
O Apache Spark possui consoles iterativos nas linguagens Python e Scala e seus \textit{Jobs} podem ser submetidos em batch também em Java. A API do Spark é acessada a partir de um objeto central denominado \texttt{ SparkContext }. Esse objeto contém a conexão com uma instância do cluster e a partir dele todos os outros objetos e métodos são acessados. Ao se iniciar um console do Spark, o objeto já está automaticamente disponível através da variável \texttt{ sc }. Para compreender a simplicidade do modelo de programação do Spark, o trecho de código abaixo lê um arquivo txt e faz a contagem de ocorrência das palavras.

\begin{lstlisting}[style=MyPythonStyle]
#produz um RDD onde cada item e uma linha do arquivo texto
arquivo = sc.textFile('hdfs://servidor:10001/arquivo.txt') 

#para cada linha produz N itens no novo RDD, uma para cada palavra.
palavras = arquivo.flatMap(lambda linha : linha.split(' ')) 

#cria novo RDD com tuplas do formato (palavra, 1)
palavrasCV = palavras.map(lambda palavra : (palavra, 1)) 

#executa o Reduce usando a funcaoo add para os valores das tuplas.
contagemPalafras = palavrasCV.reduceByKey(add) 

#somente nesse comando toda a computacaoo e feita de forma otimizada.
contagemPalavras.collect() 
\end{lstlisting}

Para resolver o problema de contagem de triângulos utilizando o Spark, a idéia é semelhante à usada com o modelo do MapReduce.

\subsection{Spark SQL}
O Spark SQL é um módulo do Apache Spark para processamento de dados estruturados (relacionais). Ele utiliza uma abstração denominada DataFrame, e também serve como uma máquina de execução de consultas distribuídas baseada em SQL. 

Um DataFrame do Spark SQL tem as mesmas características de um DataFrame em R ou em Pandas (Python) e pode ser criado baseando-se em diversas fontes de dados, como um arquivo json, um arquivo texto, um RDD do Spark, uma tabela do Hive, ou qualquer fonte que possua um driver JDBC. O \textit{schema} de um DataFrame pode ser inferido através de reflexão ou definido programaticamente. 

O trecho de código abaixo mostra como executar a contagem de triângulos utilizando o SQL proposto na seção \ref{sec:triangulos}.

\begin{lstlisting}[style=MyPythonStyle]
# Importa os modulos e cria o contexto
from pyspark.sql import SQLContext, Row
sqlContext = SQLContext(sc)

# Carrega o arquivo de arestas para um RDD
linhas = sc.textFile("hdfs://servidor:10001/data/triangles/twitter_combined.txt")
partes = linhas.map(lambda l: l.split())
arestas = partes.map(lambda p: Row(follower=int(p[0]), followee=int(p[1])))

# Infere o schema e registra o DataFrame como tabela
schemaWiki = sqlContext.createDataFrame(arestas)
schemaWiki.registerTempTable("arestas")

# Executa o SQL
triangulos = sqlContext.sql("SELECT count(*) FROM arestas R, arestas S, arestas T " +
    " WHERE R.follower = S.followee AND S.follower = T.followee AND T.follower = R.followee")
print triangulos.collect()
\end{lstlisting}


\subsection{Spark GraphX}
O Spark GraphX é um novo módulo do Apache Spark que fornece um conjunto de abstrações e ferramentas para processamento paralelo de grafos. Nas abstrações de nós e arestas é possível incluir propriedades, como pesos, capacidades máximas e mínimas, ou qualquer outra propriedade que seja relevante para modelar um problema. 

O Spark GraphX possui um conjunto de operações essenciais para diversos algoritmos de análise de grafos, como operações em paralelo sobre os nós/arcos dos grafos, obtenção de subgrafos, inversão de arestas e agregação de vizinhos. Esse último, por exemplo, pode ser usado para calcular o grau de cada vértice de um grafo. Mais detalhes desse módulo podem ser obtidos na documentação oficial do produto \footnote{http://spark.apache.org/docs/latest/graphx-programming-guide.html}. 

Como é recente seu desenvolvimento, ainda são poucos os algoritmos implementados e a única linguagem suportada é o Scala. Atualmente conta com algoritmos de PageRank, Identificação de componentes conectados e contagem de triângulos, que demonstramos o uso no trecho de código abaixo. Comparado com as estratégias de MapReduce, a contagem de triângulos utilizando Spark GraphX é de várias ordens de grandeza mais rápido e eficiente em consumo de memória. 

\begin{lstlisting}[style=MyPythonStyle]
# Carrega o grafo a partir de um arquivo cujas linhas sao pares de identificadores dos nos, definindo uma aresta
val graph = GraphLoader.edgeListFile(sc, "graphx/data/followers.txt", true).partitionBy(PartitionStrategy.RandomVertexCut)

# conta o numero de triangulos
val triCounts = graph.triangleCount().vertices

# imprime o resultado
println(triCounts.mkString("\n"))
\end{lstlisting}

\section{Considerações finais}
\label{sec:final}

Neste capítulo foram apresentadas algumas tecnologias recentes da era do big data. Entretanto, 
a lista de tecnologias com certeza não foi exaustiva. Não foram abordadas muitas outras
tecnolgias existentes. Como exemplo, não foram citadas as tecnologias de bancos de dados
não relacionais (NoSQL) \citep{han2011survey}, e a comparação entre estas e 
o modelo relacional \citep{cattell2011scalable, stonebraker2010sql, stonebraker2012newsql}.
Não é objetivo deste capítulo prover um \textit{survey} sobre todas as
possíveis tecnologias de gerenciamento de big data, mas de mostrar de forma introdutória
algumas das principais tecnologias já maduras de mercado.

O que também está fora do escopo deste capítulo é a comparação técnica mais profunda e 
de desempenho entre cada uma das tecnologias \citep{Pavlo2009}, como por exemplo, comparação entre 
bancos de dados orientados a colunas ou a linhas \citep{Abadi2008} e a comparação entre
modelos de paralelismo Hadoop ou bancos de dados \citep{Stonebraker2010}. 



\bibliographystyle{sbc}
\bibliography{bibliografia}

\end{document}
