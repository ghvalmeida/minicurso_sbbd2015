\begin{abstract}
\begin{otherlanguage}{english}
This chapter is an overview of big data management, and
intends to explore and diferentiate several recent technologies. A
classic problem of the database community will be used as a background for the 
examples given throughout this course: triangle counting on graphs. This 
problem has been chosen because it is being extensively used to identify 
the importance of individuals on social networks. 
Also, since it can be described by an algorithm that is simple to understand and 
yet complex to execute in terms of performance, the differences between technologies
in design and performance will be easily demonstrated. 
\end{otherlanguage}
\end{abstract}

\begin{resumo}
Este capítulo pretende explorar e diferenciar de forma introdutória diversas 
tecnologias recentes
para gerenciamento de dados na era do big data. Será utilizado como pano de fundo para os
exemplos um problema clássico da comunidade de bancos de dados: a contagem de triângulos
em grafos. Ele foi escolhido por ser um problema atual e prático, frequentemente 
utilizado para identificar a importância de indivíduos em redes sociais. Além disso, ele é de
fácil representação e alta complexidade de execução. Através do seu uso, é possível demonstrar
as diferenças entre as tecnologias em termos de expressividade e desempenho. 
\end{resumo}

