\section{Bancos de dados paralelos}

Bancos de dados paralelos ou massivamente paralelos (MPP, do inglês 
\emph{massively parallel processing}) também não são tecnologia 
recente, discussões sobre esse assunto e implementações de sistemas
datam da década de 80-90 \cite{Dewitt1992, Fushimi1986}. Nestes 
sistemas, os dados das tabelas são distribuídos em diversos 
nós de um cluster e o processamento de consulta é paralelizado.

Os dados podem ser distribuídos através de particionamento vertical,
onde as tabelas são quebradas por colunas; horizontal, onde as 
tabelas são quebradas por tuplas; ou ambos. A distribuição dos 
dados entre os nós pode ou não utilizar o conhecimento prévio
das consultas mais utilizas no sistema (\emph{query workload}),
e o otimizador de consultas, nestes casos, conhecendo a forma
como os dados estão distribuídos, pode repassar a parte das
consultas para os nós específicos que devem processá-la.

Há diversas formas de distribuição dos dados entre os nós, mas 
a mais comum e implementada pela grande maioria dos bancos de 
dados paralelos é 

Shared-nothing \cite{Stonebraker1986}

Aqui vamos falar sobre bancos de dados paralelos.
